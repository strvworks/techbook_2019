\section*{まえがき}
こんにちは.
Jumpaku(\ref{fig:jumpakuJ})と申します.
本作品は主人公である\ruby[g]{愛}{あい}が仲間の死の真相を追い求める説法系推理ノベルゲームです.
本作品は「第9回LOCAL学生部総大会 ヤバい同人誌執筆しようぜ」というイベントにおいて,
情報技術系に関連した同人誌の記事として執筆されました.
情報技術として,\LaTeX によってノベルゲームを作成すること,
プログラミングによって効率的に論理クイズを解くこと
がコンセプトとなっています.
同時に,本作品は説法系推理アドベンチャシリーズ(
\href{http://jumpaku.hatenablog.com/entry/2016/04/14/002437}{\underline{愛と血の\ruby[g]{修羅場}{サスペンス}}},
\href{http://jumpaku.hatenablog.com/entry/2016/07/24/032632}{\underline{恋と友情の\ruby[g]{常識}{ファイト}}}および
\href{http://jumpaku.hatenablog.com/entry/2017/07/24/044918}{\underline{罪と幸せの\ruby[g]{四苦八苦}{ノアズアーク}}})の外伝ともなっています.

本作品はPDFファイルとして頒布されることを前提としています.
遊び方としては,
プレイヤは基本的にストーリーを読み進め,途中で選択肢がある時は,選択したリンク先へ飛んでください.

\section{愛の仲間}\label{section:jumpakubegin}
私の名は\ruby[g]{愛}{あい}(\ref{fig:jumpakuI}).
探偵である.
私には8人の仲間
\ruby[g]{恵伊}{えい}(\ref{fig:jumpakuA}),
\ruby[g]{美衣}{びい}(\ref{fig:jumpakuB}),
\ruby[g]{史衣}{しい}(\ref{fig:jumpakuC}),
\ruby[g]{出井}{でい}(\ref{fig:jumpakuD}),
\ruby[g]{良威}{いい}(\ref{fig:jumpakuE}),
\ruby[g]{恵夫}{えふ}(\ref{fig:jumpakuF}),
\ruby[g]{寺井}{じい}(\ref{fig:jumpakuG}),
\ruby[g]{永一}{えいいち}(\ref{fig:jumpakuH})がいた.
\begin{figure}[b]\centering
\begin{tabular}{cccc}
\begin{minipage}{0.2\textwidth}\includegraphics[width=\textwidth]{./jumpakuasset/A.png}\caption{恵伊}\label{fig:jumpakuA}\end{minipage}
\begin{minipage}{0.2\textwidth}\includegraphics[width=\textwidth]{./jumpakuasset/B.png}\caption{美衣}\label{fig:jumpakuB}\end{minipage}
\begin{minipage}{0.2\textwidth}\includegraphics[width=\textwidth]{./jumpakuasset/C.png}\caption{史衣}\label{fig:jumpakuC}\end{minipage}
\begin{minipage}{0.2\textwidth}\includegraphics[width=\textwidth]{./jumpakuasset/D.png}\caption{出井}\label{fig:jumpakuD}\end{minipage}\\
\begin{minipage}{0.2\textwidth}\includegraphics[width=\textwidth]{./jumpakuasset/E.png}\caption{良威}\label{fig:jumpakuE}\end{minipage}
\begin{minipage}{0.2\textwidth}\includegraphics[width=\textwidth]{./jumpakuasset/F.png}\caption{恵夫}\label{fig:jumpakuF}\end{minipage}
\begin{minipage}{0.2\textwidth}\includegraphics[width=\textwidth]{./jumpakuasset/G.png}\caption{寺井}\label{fig:jumpakuG}\end{minipage}
\begin{minipage}{0.2\textwidth}\includegraphics[width=\textwidth]{./jumpakuasset/B.png}\caption{永一}\label{fig:jumpakuH}\end{minipage}\\
\begin{minipage}{0.2\textwidth}\includegraphics[width=\textwidth]{./jumpakuasset/Me.png}\caption{愛}\label{fig:jumpakuI}\end{minipage}
\begin{minipage}{0.2\textwidth}\includegraphics[width=\textwidth]{./jumpakuasset/JumpakuMark_250x250.png}\caption{Jumpaku}\label{fig:jumpakuJ}\end{minipage}
\end{tabular}
\end{figure}
彼らの詳しい過去についてはここでは省略するが
\footnote{\href{http://jumpaku.hatenablog.com/entry/2016/04/14/002437}{\underline{愛と血の\ruby[g]{修羅場}{サスペンス}}},
\href{http://jumpaku.hatenablog.com/entry/2016/07/24/032632}{\underline{恋と友情の\ruby[g]{常識}{ファイト}}}および
\href{http://jumpaku.hatenablog.com/entry/2017/07/24/044918}{\underline{罪と幸せの\ruby[g]{四苦八苦}{ノアズアーク}}}をプレイしてください.},
既にこのうち5人が死んでいる.
まず,寺井が恵夫に殺された.
次に,良威が恵伊に殺され,私は報復した.
さらに,恵夫が史衣に殺され,私はまた報復した.
残ったのは,私,美衣,出井,永一の4人だけだった.

美衣,出井,永一の関係を簡単に紹介しておく.
美衣と永一は2人の人格で1つの体を共有しているが,私は2人として扱っている.
また,美衣と永一は姉弟である.
そして,出井と美衣はこの前結婚した.
彼らは生きる希望に満ち溢れていた.

\section{愛の事件}
それはまさに不可能犯罪だった.
密室にいた美衣,出井,永一の3人が消された.
跡形もなく消滅していた.
どのようにそれがなされたのかは分からないし,解明されることも無いだろう.
現場に残されたのはは血文字で書かれた不完全なダイイングメッセージだけだった.

「犯人は...」

\section{愛の決意}
事実は小説より奇なり.
密室より人間が消失した.
現実の世界でそんなことが発生するのだろうか?
いや,しない.
私が現実だと思って生きてきたこの世界は,夢,ゲームまたは小説の中の世界なのかもしれない.
それは,人間が密室から消えるという事象は現実に発生するとは考えられないからだ.
現実の世界で発生しえない事象が発生したなら,その世界は現実の世界ではない.
逆に,この世界が現実ではないとすると,現実で発生しえない事象でも発生しうる.
世界の枠組みを考え直せば,不可能なものが可能となることがあるのだ.
例えば,現実の世界という枠組みの中では人が消える事は無いが,もし小説の世界という枠組みなら人が消えることもある.
私がこれまで現実と思ってきた世界は,実は小説の世界なのかもしれない.
これを現実と確かめる事はできない.
まさに胡蝶の夢である.

しかし,真実を追い求める探偵は,世界の枠組みを考え直すだけで変わってしまうような些細なことに惑わされてはいけない.
探偵が真実を求める際に武器にできるのは論理的な推論しか無いのだ.
前提条件から論理的に推論された結論は,例えこの世界がどのような世界であっても揺らがない.
そのような結論が,探偵が追い求めるべき真実なのだ.
例え,犯人が未登場の者,中国人,探偵自身,双子,複数犯,捜査員,端役,プロの犯罪者または被害者自身であったとしても,
例え,犯行が秘密の通路もしくは隠し部屋,未発見の毒薬,難解な説明を要し,もしくは未来に登場する科学技術または超自然の術によって行われたとしても,
例え,動機が不明瞭であったり,状況に説得力が無かったとしても,
探偵は論理的な推論によって結論を導き,これを真相としなくてはいけない.

\section{愛の推理}
誰かが3人を消したのか?
それとも,3人の中の誰かが犯人で,被害者を消した後,隠れ続けているのか?
まず,私は犯行機会のあった容疑者として私自身,美衣,出井,永一の4人を挙げた.

その上で,美衣と永一が2心同体であったこと,
出井と美衣が愛し合っていたこと,
自分の行動の記憶,
その他3人の性格等を考慮した結果,以下の主張が成り立つと結論付けた.
\begin{itemize}
\item 私は犯人ではなく被害者でもない.
\item 美衣が犯人ならば永一も犯人である.
\item 美衣が被害者ならば永一も被害者である.
\item 容疑者のうち犯人であり被害者でもある者はいない.
\item 容疑者のうち犯人または被害者である者は3人である.
\item 出井は犯人ではない.
\item 美衣が犯人ならば出井は被害者ではない.
\end{itemize}

これらの主張に基づいて,私は上の容疑者のうち犯人は誰なのか,容疑者のうち被害者は誰なのか,
といった真相を論理的に導いた\footnote{ヒントとして,真実を導くプログラムの例を\ref{section:jumpakuprogram}に示す.}.
\begin{description}
\item[容疑者のうち犯人は愛であり, 被害者は出井である.] \ref{section:jumpakufailure1}へ進む.
\item[容疑者のうち犯人は愛であり, 被害者は出井,愛である.] \ref{section:jumpakufailure2}へ進む.
\item[容疑者のうち犯人は愛であり, 被害者は美衣,出井,永一である.] \ref{section:jumpakufailure3}へ進む.
\item[容疑者のうち犯人は永一であり, 被害者は愛である.] \ref{section:jumpakufailure4}へ進む.
\item[容疑者のうち犯人は永一であり, 被害者は美衣,愛である.] \ref{section:jumpakufailure5}へ進む.
\item[容疑者のうち犯人は永一,愛であり, 被害者は美衣,出井,愛である.] \ref{section:jumpakufailure6}へ進む.
\item[容疑者のうち犯人は出井であり, 被害者は愛である.] \ref{section:jumpakufailure7}へ進む.
\item[容疑者のうち犯人は出井であり, 被害者は出井,永一,愛である.] \ref{section:jumpakufailure8}へ進む.
\item[容疑者のうち犯人は出井,愛であり, 被害者は愛である.] \ref{section:jumpakufailure9}へ進む.
\item[容疑者のうち犯人は出井,永一であり, 被害者は美衣である.] \ref{section:jumpakufailure10}へ進む.
\item[容疑者のうち犯人は出井,永一,愛であり, 被害者は出井,永一,愛である.] \ref{section:jumpakufailure11}へ進む.
\item[容疑者のうち犯人は美衣,永一であり, 被害者は美衣,永一である.] \ref{section:jumpakufailure12}へ進む.
\item[容疑者のうち犯人は美衣,永一であり, 被害者は美衣,出井である.] \ref{section:jumpakufailure13}へ進む.
\item[容疑者のうち犯人は美衣,永一,愛であり, 被害者は美衣,愛である.] \ref{section:jumpakufailure14}へ進む.
\item[容疑者のうち犯人は美衣,永一,愛であり, 被害者は美衣,出井である.] \ref{section:jumpakufailure15}へ進む.
\item[容疑者のうち犯人は美衣,出井であり, 被害者は出井である.] \ref{section:jumpakufailure16}へ進む.
\item[容疑者のうち犯人は美衣,出井であり, 被害者は出井,永一である.] \ref{section:jumpakufailure17}へ進む.
\item[容疑者のうち犯人は美衣,出井,愛であり, 被害者は愛である.] \ref{section:jumpakufailure18}へ進む.
\item[容疑者のうち犯人は美衣,出井,愛であり, 被害者は美衣,出井,愛である.] \ref{section:jumpakufailure19}へ進む.
\item[容疑者のうち犯人は美衣,出井,永一であり, 被害者は美衣,出井である.] \ref{section:jumpakufailure20}へ進む.
\item[容疑者のうち犯人は美衣,出井であり, 被害者はいない.] \ref{section:jumpakufailure21}へ進む.
\item[容疑者の中に犯人はおらず,容疑者のうち被害者は美衣,出井,永一である.] \ref{section:jumpakusuccess23}へ進む.
\item[容疑者の中に犯人は美衣,出井,永一であり,容疑者のうち被害者はいない.] \ref{section:jumpakufailure24}へ進む.
\item[容疑者の中に犯人はおらず,容疑者のうち被害者もいない.] \ref{section:jumpakufailure25}へ進む.
\item[容疑者のうち犯人はおらず, 被害者は永一である.] \ref{section:jumpakufailure26}へ進む.
\item[容疑者のうち犯人はおらず, 被害者は出井,永一である.] \ref{section:jumpakufailure27}へ進む.
\end{description}
\newpage

\subsection{失敗}\label{section:jumpakufailure1}間違えた.探偵失敗だ.失敗END...\ref{section:jumpakubegin}へ戻る.\newpage
\subsection{失敗}\label{section:jumpakufailure2}間違えた.探偵失敗だ.失敗END...\ref{section:jumpakubegin}へ戻る.\newpage
\subsection{失敗}\label{section:jumpakufailure3}間違えた.探偵失敗だ.失敗END...\ref{section:jumpakubegin}へ戻る.\newpage
\subsection{失敗}\label{section:jumpakufailure4}間違えた.探偵失敗だ.失敗END...\ref{section:jumpakubegin}へ戻る.\newpage
\subsection{失敗}\label{section:jumpakufailure5}間違えた.探偵失敗だ.失敗END...\ref{section:jumpakubegin}へ戻る.\newpage
\subsection{失敗}\label{section:jumpakufailure6}間違えた.探偵失敗だ.失敗END...\ref{section:jumpakubegin}へ戻る.\newpage
\subsection{失敗}\label{section:jumpakufailure7}間違えた.探偵失敗だ.失敗END...\ref{section:jumpakubegin}へ戻る.\newpage
\subsection{失敗}\label{section:jumpakufailure8}間違えた.探偵失敗だ.失敗END...\ref{section:jumpakubegin}へ戻る.\newpage
\subsection{失敗}\label{section:jumpakufailure9}間違えた.探偵失敗だ.失敗END...\ref{section:jumpakubegin}へ戻る.\newpage
\subsection{失敗}\label{section:jumpakufailure10}間違えた.探偵失敗だ.失敗END...\ref{section:jumpakubegin}へ戻る.\newpage
\subsection{失敗}\label{section:jumpakufailure11}間違えた.探偵失敗だ.失敗END...\ref{section:jumpakubegin}へ戻る.\newpage
\subsection{失敗}\label{section:jumpakufailure12}間違えた.探偵失敗だ.失敗END...\ref{section:jumpakubegin}へ戻る.\newpage
\subsection{失敗}\label{section:jumpakufailure13}間違えた.探偵失敗だ.失敗END...\ref{section:jumpakubegin}へ戻る.\newpage
\subsection{失敗}\label{section:jumpakufailure14}間違えた.探偵失敗だ.失敗END...\ref{section:jumpakubegin}へ戻る.\newpage
\subsection{失敗}\label{section:jumpakufailure15}間違えた.探偵失敗だ.失敗END...\ref{section:jumpakubegin}へ戻る.\newpage
\subsection{失敗}\label{section:jumpakufailure16}間違えた.探偵失敗だ.失敗END...\ref{section:jumpakubegin}へ戻る.\newpage
\subsection{失敗}\label{section:jumpakufailure17}間違えた.探偵失敗だ.失敗END...\ref{section:jumpakubegin}へ戻る.\newpage
\subsection{失敗}\label{section:jumpakufailure18}間違えた.探偵失敗だ.失敗END...\ref{section:jumpakubegin}へ戻る.\newpage
\subsection{失敗}\label{section:jumpakufailure19}間違えた.探偵失敗だ.失敗END...\ref{section:jumpakubegin}へ戻る.\newpage
\subsection{失敗}\label{section:jumpakufailure20}間違えた.探偵失敗だ.失敗END...\ref{section:jumpakubegin}へ戻る.\newpage
\subsection{失敗}\label{section:jumpakufailure21}間違えた.探偵失敗だ.失敗END...\ref{section:jumpakubegin}へ戻る.\newpage
\subsection{失敗}\label{section:jumpakufailure22}間違えた.探偵失敗だ.失敗END...\ref{section:jumpakubegin}へ戻る.\newpage
\subsection{成功}\label{section:jumpakusuccess23}真実へ到達した.\ref{section:jumpakuresult}へ進む.\newpage
\subsection{失敗}\label{section:jumpakufailure24}間違えた.探偵失敗だ.失敗END...\ref{section:jumpakubegin}へ戻る.\newpage
\subsection{失敗}\label{section:jumpakufailure25}間違えた.探偵失敗だ.失敗END...\ref{section:jumpakubegin}へ戻る.\newpage
\subsection{失敗}\label{section:jumpakufailure26}間違えた.探偵失敗だ.失敗END...\ref{section:jumpakubegin}へ戻る.\newpage
\subsection{失敗}\label{section:jumpakufailure27}間違えた.探偵失敗だ.失敗END...\ref{section:jumpakubegin}へ戻る.\newpage

\section{愛の真相}\label{section:jumpakuresult}
真犯人は容疑者の中にはいない.
美衣,出井,永一は全員被害者だ.
これは論理的に導かれるたった一つの真相である.
例え,どんな理由があろうとも,
論理的な推論によって導かれる結論なのだから,これは真実なのだ.

\section{愛の飛躍}
探偵としての仕事が終わった.
ところが,私の思考は続いた.
失った仲間を思うと思考せずにはいられなかった.
3人を消した犯人は容疑者の中にはいない.
では,それ以外の誰なのか?
その思考は論理的な推論を飛躍した.

そもそも,
\begin{description}
\item[この世界は現実ではない.] \ref{subsection:jumpakulast1success}へ進む.
\item[この世界は現実である.] \ref{subsection:jumpakulast1failure}へ進む.
\end{description}
\newpage

\subsection{成功}\label{subsection:jumpakulast1success}なぜなら,密室から人間が消失する事は現実の世界では発生しないのだから.
次に,現実ではないこの世界は
\begin{description}
\item[何者かの意思によって作り出されている気がする.] \ref{subsection:jumpakulast2success}へ進む.
\item[ランダムに生成されている気がする.] \ref{subsection:jumpakulast2failure}へ進む.
\end{description}
\newpage

\subsection{失敗}\label{subsection:jumpakulast1failure}\ref{section:jumpakubegin}へ戻る.\newpage

\subsection{成功}\label{subsection:jumpakulast2success}
なぜなら,私の周りで殺人事件が多すぎること,
私を含めた仲間の名前がアルファベット順に並ぶことなど,不自然なことが多いから.

この世界を作り,私の仲間を消した犯人の名前の頭文字は
\begin{description}
\item[A] \ref{subsection:jumpakulast3failureA}へ進む.
\item[B] \ref{subsection:jumpakulast3failureB}へ進む.
\item[C] \ref{subsection:jumpakulast3failureC}へ進む.
\item[D] \ref{subsection:jumpakulast3failureD}へ進む.
\item[E] \ref{subsection:jumpakulast3failureE}へ進む.
\item[F] \ref{subsection:jumpakulast3failureF}へ進む.
\item[G] \ref{subsection:jumpakulast3failureG}へ進む.
\item[H] \ref{subsection:jumpakulast3failureH}へ進む.
\item[I] \ref{subsection:jumpakulast3failureI}へ進む.
\item[J] \ref{subsection:jumpakulast3successJ}へ進む.
\item[K] \ref{subsection:jumpakulast3failureK}へ進む.
\item[L] \ref{subsection:jumpakulast3failureL}へ進む.
\item[M] \ref{subsection:jumpakulast3failureM}へ進む.
\item[N] \ref{subsection:jumpakulast3failureN}へ進む.
\item[O] \ref{subsection:jumpakulast3failureO}へ進む.
\item[P] \ref{subsection:jumpakulast3failureP}へ進む.
\item[Q] \ref{subsection:jumpakulast3failureQ}へ進む.
\item[R] \ref{subsection:jumpakulast3failureR}へ進む.
\item[S] \ref{subsection:jumpakulast3failureS}へ進む.
\item[T] \ref{subsection:jumpakulast3failureT}へ進む.
\item[U] \ref{subsection:jumpakulast3failureU}へ進む.
\item[V] \ref{subsection:jumpakulast3failureV}へ進む.
\item[W] \ref{subsection:jumpakulast3failureW}へ進む.
\item[X] \ref{subsection:jumpakulast3failureX}へ進む.
\item[Y] \ref{subsection:jumpakulast3failureY}へ進む.
\item[Z] \ref{subsection:jumpakulast3failureZ}へ進む.
\end{description}
\newpage

\subsection{失敗}\label{subsection:jumpakulast2failure}\ref{section:jumpakubegin}へ戻る.\newpage

\subsection{失敗}\label{subsection:jumpakulast3failureA}\ref{section:jumpakubegin}へ戻る.\newpage
\subsection{失敗}\label{subsection:jumpakulast3failureB}\ref{section:jumpakubegin}へ戻る.\newpage
\subsection{失敗}\label{subsection:jumpakulast3failureC}\ref{section:jumpakubegin}へ戻る.\newpage
\subsection{失敗}\label{subsection:jumpakulast3failureD}\ref{section:jumpakubegin}へ戻る.\newpage
\subsection{失敗}\label{subsection:jumpakulast3failureE}\ref{section:jumpakubegin}へ戻る.\newpage
\subsection{失敗}\label{subsection:jumpakulast3failureF}\ref{section:jumpakubegin}へ戻る.\newpage
\subsection{失敗}\label{subsection:jumpakulast3failureG}\ref{section:jumpakubegin}へ戻る.\newpage
\subsection{失敗}\label{subsection:jumpakulast3failureH}\ref{section:jumpakubegin}へ戻る.\newpage
\subsection{失敗}\label{subsection:jumpakulast3failureI}\ref{section:jumpakubegin}へ戻る.\newpage
\subsection{成功}\label{subsection:jumpakulast3successJ}\ref{section:jumpakuending}へ進む.\newpage
\subsection{失敗}\label{subsection:jumpakulast3failureK}\ref{section:jumpakubegin}へ戻る.\newpage
\subsection{失敗}\label{subsection:jumpakulast3failureL}\ref{section:jumpakubegin}へ戻る.\newpage
\subsection{失敗}\label{subsection:jumpakulast3failureM}\ref{section:jumpakubegin}へ戻る.\newpage
\subsection{失敗}\label{subsection:jumpakulast3failureN}\ref{section:jumpakubegin}へ戻る.\newpage
\subsection{失敗}\label{subsection:jumpakulast3failureO}\ref{section:jumpakubegin}へ戻る.\newpage
\subsection{失敗}\label{subsection:jumpakulast3failureP}\ref{section:jumpakubegin}へ戻る.\newpage
\subsection{失敗}\label{subsection:jumpakulast3failureQ}\ref{section:jumpakubegin}へ戻る.\newpage
\subsection{失敗}\label{subsection:jumpakulast3failureR}\ref{section:jumpakubegin}へ戻る.\newpage
\subsection{失敗}\label{subsection:jumpakulast3failureS}\ref{section:jumpakubegin}へ戻る.\newpage
\subsection{失敗}\label{subsection:jumpakulast3failureT}\ref{section:jumpakubegin}へ戻る.\newpage
\subsection{失敗}\label{subsection:jumpakulast3failureU}\ref{section:jumpakubegin}へ戻る.\newpage
\subsection{失敗}\label{subsection:jumpakulast3failureV}\ref{section:jumpakubegin}へ戻る.\newpage
\subsection{失敗}\label{subsection:jumpakulast3failureW}\ref{section:jumpakubegin}へ戻る.\newpage
\subsection{失敗}\label{subsection:jumpakulast3failureX}\ref{section:jumpakubegin}へ戻る.\newpage
\subsection{失敗}\label{subsection:jumpakulast3failureY}\ref{section:jumpakubegin}へ戻る.\newpage
\subsection{失敗}\label{subsection:jumpakulast3failureZ}\ref{section:jumpakubegin}へ戻る.\newpage

\section{愛の終点}\label{section:jumpakuending}
私の世界を作り,私を生かし,仲間を奪ったのはこの物語を書いたJumpakuである.
動機はきっと「面白いと思ったから」だろう.
私は世界の理を理解した.

そして.

それでも.

もっと,生きよう.

解決END.
\newpage

\section*{サンプルコード}\label{section:jumpakuprogram}
以下に今回の事件の論理的推論を行うためのPython3のプログラムを示す.
\inputminted[linenos]{python3}{./jumpakuasset/LogicQuiz.py}

\section*{あとがき}
本章にはBSD 2-Clause Licenseが適用されます.

\begin{quote}
BSD 2-Clause License

Copyright (c) 2018, Jumpaku
All rights reserved.

Redistribution and use in source and binary forms, with or without
modification, are permitted provided that the following conditions are met:
\begin{itemize}
\item Redistributions of source code must retain the above copyright notice, this
  list of conditions and the following disclaimer.
\item Redistributions in binary form must reproduce the above copyright notice,
  this list of conditions and the following disclaimer in the documentation
  and/or other materials provided with the distribution.
\end{itemize}
THIS SOFTWARE IS PROVIDED BY THE COPYRIGHT HOLDERS AND CONTRIBUTORS "AS IS"
AND ANY EXPRESS OR IMPLIED WARRANTIES, INCLUDING, BUT NOT LIMITED TO, THE
IMPLIED WARRANTIES OF MERCHANTABILITY AND FITNESS FOR A PARTICULAR PURPOSE ARE
DISCLAIMED. IN NO EVENT SHALL THE COPYRIGHT HOLDER OR CONTRIBUTORS BE LIABLE
FOR ANY DIRECT, INDIRECT, INCIDENTAL, SPECIAL, EXEMPLARY, OR CONSEQUENTIAL
DAMAGES (INCLUDING, BUT NOT LIMITED TO, PROCUREMENT OF SUBSTITUTE GOODS OR
SERVICES; LOSS OF USE, DATA, OR PROFITS; OR BUSINESS INTERRUPTION) HOWEVER
CAUSED AND ON ANY THEORY OF LIABILITY, WHETHER IN CONTRACT, STRICT LIABILITY,
OR TORT (INCLUDING NEGLIGENCE OR OTHERWISE) ARISING IN ANY WAY OUT OF THE USE
OF THIS SOFTWARE, EVEN IF ADVISED OF THE POSSIBILITY OF SUCH DAMAGE.
\end{quote}

また,キャラクタの画像は罪と幸せの\ruby[g]{四苦八苦}{ノアズアーク}より引用しました.