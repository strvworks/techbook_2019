\section{はじめに}
この章では,同人誌を出すと言われて果たして何を書いたものかと悩んだ情報系学生が乱心し,全く触れたことのないファミコンエミュレータを開発してみた話を書かせていただきます.
ファミコンエミュレータの開発を通して,CPUの動作やリソースの少ない環境での工夫などに触れて行くことを目的としてやっていこうと思います.
色んなサイトから情報を集めつつ書いていきますので,誤った点があるかもしれませんがご了承ください.

\section{エミュレータとは}
一口にエミュレータというと様々なエミュレータが存在しますが,ここで言うエミュレータとは,何らかの機能を持ったコンピュータやハードウェアの動作を模倣するソフトウェアのことです.
コンピュータ分野で馴染みの深いエミュレータとしては,既に起動しているOSの上で他のOSなどのソフトウェアを起動するための仮想環境を提供するソフトウェアなどがあります.

\section{エミュレータの方式}
エミュレータの実行方式には大きく分けて2つの方式が存在します.

まず1つの方式が,エミュレータが動作するハードウェアの実CPUの計算能力の一部をそのまま利用する方式です.この方式ではエミュレータ上で動作させるソフトウェアへのオーバーヘッドが少なく,高速な動作が可能です.
しかし,この方式では動作させたいソフトウェアが想定するアーキテクチャが利用する実ハードウェアと一致していなければならないというデメリットがあります.
この方式にはVM VirtualBoxやVMWare Playerなどが該当し,これらのエミュレータの上では,前述の理由からARMやRICS-Vなどに向けたソフトウェアが動作しません.

もう一つの方式として,特定のアーキテクチャのCPUの動作をソフトウェア上で模倣し,その上で動作させたいソフトウェアをエミュレートする方式があります.
こちらの方式では,エミュレートしたいCPUのアーキテクチャに従って実装を行えば実ハードウェアのアーキテクチャは関係がないため,より多くのソフトウェアをエミュレートすることができます.
しかし,実ハードウェアの上で別のCPUの動作をするソフトウェアを動かすということになるため,エミュレートによるオーバーヘッドが大きくなります.
この方式にはQEMUなどが該当し,Android Studioなどに搭載されているAndroidエミュレータもQEMUをベースに開発がなされているため,ARMアーキテクチャ向けであるAndroidをx86環境上で動作させることができています.\footnote{Android-x86などのx86移植版はVirtualBoxやVMWareでも動作しますが,一部のAndroidアプリケーションの動作に違いが出ることがあります.}
また,こちらの方式には高レベルエミュレーションと低レベルエミュレーションと呼ばれる更に細かい2つの分類が存在します.
それぞれ,エミュレートするCPUをハードウェアレベルからソフトウェアで実装する方式と,命令に対する応答のみをソフトウェアで実装する方式となっており,より精密なエミュレーションが要求される場合は前者,動作のみをエミュレートすれば良い場合は後者が選ばれます.一般的に,高レベルエミュレーションのほうが軽量に動作します.

\section{ゲームエミュレータとは}
ゲームエミュレータとは,特定のコンシューマ(コンソール)機やアーケード機などのゲームハードに向けて開発されたソフトウェアを,実機以外の環境で動作させることを目的としたエミュレータのことです.
実際にメーカーが開発したゲームソフトウェアをエミュレータで動作させる行為については色々と権利問題が複雑なためここでは触れませんが,個人がコンシューマー機などに向けて開発したオリジナルのソフトウェア(ROM)も多く存在し,もちろんそれらは安全にエミュレータ上で動作させることが可能です.
自分でROMを作成したオリジナルのゲームを名のしれたゲームハードと同じ環境で動かすことができるのってなかなか夢がありませんか?

また,任天堂のWiiに搭載されているバーチャルコンソールなどは,過去に自社ハードで発売したゲームをWiiで動作させる公式のゲームエミュレータとして開発されたものです.

\section{ファミコンの構成}
今回の開発では,ファミコンことファミリーコンピュータのエミュレータを開発してみますが,その前にまずはファミコンがどのような構成で動作しているのか確認しましょう.なお,ここで扱う構成は日本で発売されたファミコンを対象とし,更にディスクシステムやネットワークシステムについては扱いません.

\subsection{全体の概要}
まずは,全体の概要を見てみます.表\ref{famicom-spec}におおまかなファミコンのスペックをまとめてみました.
\begin{table}[h]
\centering
\caption{スペック概要}
\label{famicom-spec}
\begin{tabular}{|l|l|l|}
\hline
\multirow{3}{*}{CPU} & チップ & Ricoh 2A03 (MOS 6502カスタム) \\ \cline{2-3}
 & アドレス空間 & 64KB (\$0000 - \$FFFF) \\ \cline{2-3}
 & クロック & 1.789773 MHz \\ \hline
Main Memory & ワーキングRAM & 2KB \\ \hline
\multirow{5}{*}{Graphics} & チップ & Ricoh 2C02 \\ \cline{2-3}
 & ビデオRAM & 2KB \\ \cline{2-3}
 & 解像度 & 256x240 (W x H) オーバースキャン非考慮 \\ \cline{2-3}
 & 色数 & 52色 (13色x4階調) \\ \cline{2-3}
 & 同時色数 & \begin{tabular}[c]{@{}l@{}}BG色13色(汎用12色+背景色),\\ スプライト12色(汎用12色,背景は透明な為無し),\\ 計25色\end{tabular} \\ \hline
\multirow{2}{*}{Sounds} & チップ & Ricoh 2A03 内蔵機能 \\ \cline{2-3}
 &  & あとでかく \\ \hline
\multirow{2}{*}{カートリッジ} &  & あとでかく \\ \cline{2-3}
 &  &  \\ \hline
\end{tabular}
\end{table}


\section{ファミコンのスペック}
ファミコンの各種スペックを以下に示します.
\begin{minted}[frame=lines,framesep=2mm,baselinestretch=1.2,fontsize=\footnotesize,linenos,breaklines]{text}
\end{minted}

\section{検証環境}
今回開発を行うにあたって使用した環境を以下に示します.
\begin{minted}[frame=lines,framesep=2mm,baselinestretch=1.2,fontsize=\footnotesize,linenos,breaklines]{text}
    Archlinux (x86_64 5.0.3.arch1-1)
    go 1.12 linux/amd64
    sdl 2.0.9
\end{minted}

