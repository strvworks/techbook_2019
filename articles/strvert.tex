\section{はじめに}
この章では,同人誌を出すと言われて果たして何を書いたものかと悩んだ情報系学生が乱心し,全く触れたことのないゲームエミュレータを開発してみた話を書かせていただきます.

\section{エミュレータとは}
一口にエミュレータというと様々なエミュレータが存在しますが,ここで言うエミュレータとは,何らかの機能を持ったコンピュータやハードウェアの動作を模倣するソフトウェアのことです.
コンピュータ分野で馴染みの深いエミュレータとしては,既に起動しているOSの上で他のOSなどのソフトウェアを起動するための仮想環境を提供するソフトウェアなどがあります.

\section{エミュレータの方式}
エミュレータの実行方式には大きく分けて2つの方式が存在します.

まず1つの方式が,エミュレータが動作するハードウェアの実CPUの計算能力の一部をそのまま利用する方式です.この方式ではエミュレータ上で動作させるソフトウェアへのオーバーヘッドが少なく,高速な動作が可能です.しかし,この方式では動作させたいソフトウェアが想定するアーキテクチャが利用する実ハードウェアと一致していなければならないというデメリットがあります.この方式にはVM VirtualBoxやVMWare Playerなどが該当し,これらのエミュレータの上ではARMやRICS-Vなどに向けたソフトウェアは動作しません.

もう一つの方式として,特定のアーキテクチャのCPUの動作をソフトウェア上で模倣し,その上で動作させたいソフトウェアをエミュレートする方式があります.こちらの方式では,CPUアーキテクチャごとにエミュレータを実装すれば実ハードウェアのアーキテクチャは関係がないため,より多くのソフトウェアをエミュレートすることができます.しかし,実ハードウェアの上で別のCPUの動作をするソフトウェアを動かすということになるため,エミュレートによるオーバーヘッドが大きくなります.この方式にはQEMUなどが該当し,Android Studioなどに搭載されているAndroidエミュレータもこのQEMUをベースに開発がなされているため,ARMアーキテクチャ向けであるAndroidをx86環境上で動作させることができています.(Android-x86などのx86移植版はVirtualBoxなどでも動作します.)

\section{ゲームエミュレータ}
ゲームエミュレータとは,特定のコンシューマ(コンソール)機やアーケード機に向けて開発されたゲームソフトウェアを,実機以外の環境で動作させることを目的としたエミュレータのことです.
実際にメーカーが開発したゲームソフトウェアをエミュレータで動作させる行為については色々と権利問題が複雑なためここでは触れませんが,個人がコンシューマー機などに向けて開発したオリジナルのソフトウェア(ROM)も多く存在し,もちろんそれらは安全にエミュレータ上で動作させることが可能です.自分でROMを作成してオリジナルのゲームを名のしれたゲームハードと同じ環境で動かすこともできるため,なかなか夢があります.

また,任天堂のWiiに搭載されているバーチャルコンソールなどは,過去に自社ハードで発売したゲームをWiiで動作させる公式のゲームエミュレータとして開発されたものです.

\section{検証環境}
今回開発を行うにあたって使用した環境を以下に示します.
\begin{minted}[frame=lines,framesep=2mm,baselinestretch=1.2,fontsize=\footnotesize,linenos,breaklines]{text}
    Archlinux (x86_64 5.0.3.arch1-1)
    go 1.12 linux/amd64
    sdl 2.0.9
\end{minted}

