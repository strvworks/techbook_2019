\section{はじめに}
どうも、みなさんおなじみのあるねこさんです。この記事ではPythonの型アノテーション機能を
活用し、型の力を借りつつPython上でHaskellっぽい型クラス的なモデルを再現する方法を紹介します。また実際にPythonの型アノテーション機能を使う上でのメリットや注意点をその都度解説します。それでは型の世界へレッツゴー!

\section{Pythonと型}
あまりなじみがないかもしれませんが、Python3系では関数に型アノテーションを書くことができます。その機能はバージョンアップごとに強化され、特にPython3.5では引数に型アノテーションを書けるようになり、3.6では個別の変数にも書くことができるようになりました。また、ジェネリックや高階関数などもサポートされていて、多彩な型を表現できるようになっています。

ただしこれらの型アノテーションは実行時には完全に無視され、いかなる値が渡ってきてもチェックは行われません。あくまで人間や静的コード解析ツールにとってわかりやすくなるためだけの機能となっています。そのため、以下のコードは何のエラーや警告も出さずに実行できます。

\inputpython{./assets/aruneko/sample01.py}

型チェックは専用のツールであるmypyを使って行います。あるいは、Pycharmにデフォルトで搭載されている型チェック機能を使うこともできます。ただしこれらのツールは同じコードを解析しても微妙に違う結果を返すので要注意です。この記事では特に断りが無い場合mypyを利用します。

mypyをインストールするにはpipコマンドを利用します。一昔前は``mypy-lang''パッケージとして登録されていましたが、現在は``mypy''パッケージに名称を変更しているので、過去の資料を参照する際は注意が必要です。

\begin{bashcode}
pip install mypy
\end{bashcode}

それではジェネリックを使って先ほどのコードを書き直し、mypyで解析してみます。

\inputpython{./assets/aruneko/sample02.py}

\begin{bashcode}
mypy sample02.py
\end{bashcode}

\section{リストを改造しよう}

\section{Maybeを作り出す}
