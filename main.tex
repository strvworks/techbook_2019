\documentclass[autodetect-engine,dvipdfmx-if-dvi,ja=standard,b5paper,10.5pt,twoside,openany,layout=v2]{bxjsbook}

\newcommand{\stypath}{./sty}
\newcommand{\articlepath}{./articles}
\newcommand{\assetspath}{./assets}

\newcommand{\lrfasset}{\assetspath/lrf141asset}
\newcommand{\chikuwaitasset}{\assetspath/chikuwa_ITasset/gray}
\newcommand{\strvertasset}{\assetspath/strvertasset}
\newcommand{\haibaraaaaaaaaset}{\assetpath/haibaraaaaaaaaset}
\newcommand{\yuhiasset}{\assetpath/yuhiasset}
\newcommand{\materialofmouseasset}{\assetpath/materialofmouseasset}
\newcommand{\takuzooasset}{\assetpath/takuzoo3868asset}

\usepackage{\stypath/localst17}
\usepackage{\stypath/mymintedsetting}

\usepackage{lipsum}
\usepackage{layout}

%keisuke495500
\usepackage{caption}

%Takuzoo3868
%\usepackage{dirtree}

%Jumpaku
\usepackage{pxrubrica}
\usepackage{hyperref}
\usepackage{pxjahyper}
\usepackage{comment}
\usepackage{verbatim}

%materialofmouse
%\usepackage{siunitx}

\usepackage{amsmath}
\usepackage{graphicx}

%strvert
\usepackage{multirow}

\title{情報ボーイズの寄稿ノート}
\author{うっひょい \and ちくうぇいと \and あわあわ \and けんつ \and さわだ \and Jumpaku \and あるねこ}

\date{}

\begin{document}
\frontmatter
\maketitle
\begin{myintroduce}{\chikuwaitasset/icon.jpg}{ちくうぇいと @chikuwa\_IT}
  RubyでイケイケWEBエンジニアになるつもりだったのに, 気がついたらmrubyをOSの下に組み込んだりして完全にシステムプログラミング沼に落ちてしまいました.\\
  どうしてこうなった.
\end{myintroduce}
\begin{myintroduce}{\lrfasset/icon_gray.jpg}{けんつ @lrf141}
  カーネルからWebまで色々やってます.\\
  インフラとScalaとJavaが大好物です.
\end{myintroduce}


\chapter{はじめに}
\addtolength{\oddsidemargin}{10pt}
\addtolength{\evensidemargin}{-10pt}
%\addtolength{\textwidth}{-14pt}
「情報ボーイズの寄稿ノート」を手に取っていただきありがとうございます。我々「LOCAL学生部」は、北海道でITに興味のある学生が結集して、日々技術的な話題を交換したり勉強会を開いたりしている団体です。今回はそのアウトプットの一環として、初めて同人誌という媒体を選んでみました。

メンバーそれぞれが異なる得意分野を持っているので、今回はオムニバス形式でジャンルにとらわれず自由に執筆しています。そのため電子工作やカーネルモジュールなど様々な分野にまたがる同人誌が完成しました。ぜひバラエティ豊かな記事を読んで、我々学生の熱意を感じ取っていただければと思います。それでは、さっそくページをめくってお楽しみください!


\tableofcontents
\mainmatter
%\addtolength{\oddsidemargin}{-20pt}
%\addtolength{\evensidemargin}{18pt}

\chapterauthor{chikuwait(ちくうぇいと)}
\chapter{超入門 仮想化技術}
\section{はじめに}
近年, コンテナ型仮想化の普及もあり仮想化技術という存在を様々な場面でよく聞くようになりました. しかし, コンテナ型仮想化技術以外の他の仮想化技術というのは中々個人で扱うような技術・ソフトウェアではないこともあり, あまり知られていないことが多かったりします. また, コンテナ型仮想化技術も便利で環境構築が魔法のようなすごい存在といった漠然とした解釈の人もきっと多いはずです. そこでこの章では, 各種仮想化技術の仕組みや特徴について触れながらふわっと仮想化技術について紹介します.

\section{仮想化技術とは}
仮想化技術にはいくつか種類がありますが, この章では計算機資源を抽象化してOSなどに見せるプラットフォーム仮想化のことを指します. 仮想化することで複数の計算機資源を単一に見せたり, 単一の計算機資源を複数に見せることができます. そしてプラットフォーム仮想化を支えるための技術としてシステム仮想機械(VM, Virtual Machine)と呼ばれる計算機資源をエミュレートするソフトウェアが存在します. 仮想機械の実装はハイパーバイザを始めとしていくつか存在します. 例えば仮想専用サーバ(VPS, Virtual Private Server)のようなサービスでは図2.1にように, 物理サーバ上に仮想化OSによって複数の仮想サーバに分割してユーザに各仮想サーバを提供しています.
\begin{figure}[htbp]
    \centering
    \includegraphics[width=50mm]{./assets/chikuwa_ITasset/vps.png}
    \caption{hogehoge}
    \label{fig:vps}
\end{figure}

\section{ハイパーバイザについて}
仮想化を実現するハイパーバイザにはベアメタル型(Type1),ホスト型(Type2)の2つに分類されます.

\section{ベアメタル型(Type1)}
ベアメタル型(Type1)は, ハードウェアの上で直接動作します. この方式ではホストOSと呼ばれるような土台になるOSが存在しないため,仮想マシンによる遅延や速度低下を防ぐことができます. そしてベアメタルハイパーバイザでは実装手法でもモノリシックカーネル型とマイクロカーネル型の2つに分類することができるほか, 仮想化のアプローチで完全仮想化と準仮想化に分類することができます.
\subsection{モノリシックカーネル型}
主にVMware ESX/ESXiなどで採用されている方式で, モノリシックという英語で「1枚岩」という意味の通り、ハイパーバイザの中にデバイスドライバが含まれています. ハイパーバイザがストレージをはじめ,ネットワークや入力デバイスといったハードウェアへのアクセスをすべてを処理します. この方法の利点はハイパーバイザとデバイスドライバが密接に連携するため,オーバーヘッドが少なく効率的です. しかしながら, ハイパーバイザの中にデバイスドライバが存在しているため,ハイパーバイザ層でデバイスドライバを用意する必要があります. そのため, ハードウェアのサポートがマイクロカーネル型と比較して少なく, 使用するハードウェアに制限がかかってしまう場合があります. また, デバイスドライバをハイパーバイザに直接組み込むため, バグや脆弱性はハイパーバイザ全体に広がってしまいます.
\begin{figure}[htbp]
    \centering
    \includegraphics[width=50mm]{./assets/chikuwa_ITasset/monolithic.png}
    \caption{モノリシックカーネル型ベアメタルハイパーバイザ}
    \label{fig:monolithic}
\end{figure}
\begin{figure}[htbp]
    \centering
    \includegraphics[width=50mm]{./assets/chikuwa_ITasset/monolithic.png}
    \caption{モノリシックカーネル型のハードウェアアクセス}
    \label{fig:monolithic_access}
\end{figure}
\subsection{マイクロカーネル型}
主にXenやHyper-Vで採用されている方式で, ハイパーバイザを管理する仮想マシンと管理OSを用意します. この管理OSはLinuxはWindows Serverなど汎用OSを使用します. また, 管理OSはXenではドメイン0, Hyper-Vでは親パーティションと呼ばれています. この方式では, デバイスドライバはハイパーバイザではなく, ハイパーバイザ上の仮想マシンとして動作している管理OSのデバイスドライバを使用します. 仮想マシンからハードウェアにアクセスする時はゲストOSから仮想デバイスのインターフェースを経由してハイパーバイザから管理OSに渡されます. そして管理OSのデバイスドライバからハードウェアにアクセスします.この方法は, 汎用OSのデバイスドライバを使用することで, モノリシックカーネル型に比べてハードウェアのサポートが多く, ハードウェアの対応が柔軟であるという利点があります. 例えば, Hyper-VならWindows用のデバイスドライバを使用することができます. しかしながら, ハードウェアにアクセスする際にハイパーバイザから管理OSを経由するため, モノリシックカーネル型よりも性能が低下してしまいがちであり, 管理用の汎用OSがクラッシュした場合全てのVMがクラッシュしてしまうといった欠点が存在します.
\begin{figure}[htbp]
    \centering
    \includegraphics[width=50mm]{./assets/chikuwa_ITasset/monolithic.png}
    \caption{マイクロカーネル型ベアメタルハイパーバイザ}
    \label{fig:microkernel}
\end{figure}
\begin{figure}[htbp]
    \centering
    \includegraphics[width=50mm]{./assets/chikuwa_ITasset/monolithic.png}
    \caption{マイクロカーネル型のハードウェアアクセス}
    \label{fig:microkernel_access}
\end{figure}

\subsection{完全仮想化}
完全仮想化方式のハイパーバイザでは, ハードウェアの挙動をすべてエミュレートします. そのため, 何も変更も加えていないそのままのホストOSを動かすことができます. 1960年代にIBMが「トラップアンドエミュレート」とよばれる方法で完全仮想化を実装しようとしました. この方法ではゲストOSが特権がない状態(Ring3)で実行させ, 特権(Ring0)が必要な命令を実行しようとすると失敗します. その際にハイパーバイザがその失敗をトラップして原因を確認してからその命令をエミュレートすることによってゲストOSの期待する結果を返すことができ, ゲストOSにRing0以外で実行されていることを気づかせないようにすることができます. しかしながら, この手法は古典的ですべてのアーキテクチャに適用できるわけではありませんでした. 特にx86プロセッサの場合, ユーザ権限で実行できるセンシティブ命令と呼ばれる計算機資源の構成などの依存している命令が存在しているため, 実装を難しくさせていました. そこで, 「バイナリトランスレーション」と呼ばれる新しい手法が使われるようになりました. この手法では, センシティブな命令以外の命令は直接CPUで実行し, センシティブな命令はハイパーバイザで実行前に動的に他の命令に置き換えられます.
\begin{figure}[htbp]
    \centering
    \includegraphics[width=50mm]{./assets/chikuwa_ITasset/monolithic.png}
    \caption{バイナリトランスレーション}
    \label{fig:binarytranslation}
\end{figure}
\subsection{準仮想化}
準仮想化方式のハイパーバイザでは, ハイパーバイザ上で実行するゲストOSに手を加え, 仮想環境を実現しています. この方式では, ゲストOSのカーネルが発行するハードウェアを制御するシステムコールに手を加えることでハイパーバイザと協調することで高速に動作させようとするものです. なおこのような手を加えられたシステムコールはハイパーバイザコールと呼ばれます. バイナリトランスレーションではセンシティブ命令を動的に変換していたため、性能が低下しやすい特徴がありましたが, 準仮想化では静的に変換を行い修正するため, 性能には影響はありません. しかしながら, ゲストOSに予め変更を加える必要があるため, 使用するゲストOSに制限があるという欠点があります.
\section{ホスト型(Type2)}
ホスト型(Type2)はParallels DesktopやVirtualBoxなどで採用されている方式で, LinuxやWindowsといったホストOSの上でアプリケーションとして動作します. また, ハイパーバイザをインストールする先のPCをホストOSと呼びます. 主にサーバとしての仮想化よりもユーザがmacOS上でWindowsとwindows専用ソフトウェアを使用するといったようなクライアントサイドでの用途に用いられることが多いです. この方式の利点は, ホストOSを変更することもなく, アプリケーションとしてインストールすることができるため手軽に利用することができる点です. 最近ではVargrantのようなプロビジョニングツールが登場したことにより, 手軽に開発環境として仮想環境を用意することができるようにもなりました. しかしながら,ホスト型の欠点として仮想デバイスから物理デバイスにたどり着くまでにハイパーバイザ, ホストOSを経由する必要があるため, オーバヘッドがベアメタル型に比べて大きくなってしまいます.
\begin{figure}[htbp]
    \centering
    \includegraphics[width=50mm]{./assets/chikuwa_ITasset/monolithic.png}
    \caption{ホスト型ハイパーバイザ}
    \label{fig:hosthypervisor}
\end{figure}
\section{コンテナ型仮想化}
コンテナ型仮想化はハイパーバイザによる仮想化とは少し違う存在で, 単一のOS上に「コンテナ」と呼ばれる仮想的なユーザ空間を提供しています. ここ最近Dockerと呼ばれるコンテナランタイムの普及により注目される存在となってきました. またDocker以外にもLXC/LXDやHaconiwaといった様々なコンテナランタイムが登場しています. これらのコンテナランタイムは一見するとホスト型ハイパーバイザのようにOSの上でOSを簡単に起動しているように錯覚しがちですが, あくまでもコンテナではホストOSの一つの「プロセス」であり,リソースなどを制限したり切り分けていることで一つの小さな独立した環境を作っています. ここからはLinuxコンテナランタイムに焦点を絞り, Linuxのどのような機能を使ってコンテナというものを作り上げていくかを解説していきます.
\section{Linuxコンテナランタイムをつくり上げる技術}
\subsection{Linux namespace}
Linux namespaceはOSのリソースの分離をおこなう仕組みです. Linuxの様々なリソースには「名前空間」と呼ばれるものが存在します. この名前空間を分けてあげることであたかもそのリソースしか存在しないように見せることでリソースを共存させることができます. 今回は4つの名前空間について解説します.
\subsubsection{IPC名前空間}
IPC名前空間はプロセス間通信のリソースであるSystem V IPCオブジェクトとPOSIXメッセージキュー(Linux2.6.30以降)を分離します. IPC名前空間で分離することによって名前空間が異なるプロセスが共有する共有メモリやセマフォにアクセスすることを防ぐことができます.
\subsubsection{Mount名前空間}
Mount名前空間はファイルシステムのマウントポイントを分離することで異なる名前空間のファイルシステムにアクセスを分離することができます. 通常, 子プロセスは親プロセスと同じマウントポイントを認識します. しかし新しいMount名前空間の下では, 子プロセスは任意の変更を加えることができ,親プロセスやシステム全体のMount名前空間には影響を与えません. 例えば, 各コンテナごとにMount名前空間を分けて/var, /tmpを持たせることで独立したユーザ領域を見せることができます.



\chapterauthor{はいばら}
\chapter{後で変える}
\input{\articlepath/haibaraaaaaaa}

\chapterauthor{けんつ}
\chapter{入門Linux Kernel}
\section{はじめに}
このセクションではLinux Kernelに対する知見をカーネルモジュール制作を通して深めることを目的としています。
単純なカーネルモジュールを制作してもあまり意味が無いので、ここではカーネルモジュールとして動作するEchoサーバを作ることを題材にします。

\subsection{カーネルモジュールとは}
カーネルモジュールとは、Linux上で動的につまりは起動中でも追加削除可能なモジュールを指す。
通常のOSではカーネルにモジュールを追加するとカーネルそのものを再構築する必要がでてくるが、Linuxカーネルモジュールはそれを必要とせずにモジュールの追加、利用、削除が可能となっている。
ここで紹介するカーネルモジュールはその特性上、ローダブルカーネルモジュールと呼ばれることがある。
現在ロードされているカーネルモジュールを確かめるには\mintinline{bash}{lsmod}で確認することができる。

\subsection{カーネルからHello,World!!}

\subsection{}


\chapterauthor{さわだ}
\chapter{後で変える}
\section{はじめに}
皆さんはちょっとしたメモやプログラミングにどんなエディタを使いますか?Vim?それともEmacs?まぁ色々ありますよね.
人それぞれ好き嫌いがあると思います.僕はVimを自分好みに拡張するのが好きです\footnote{そこのEmacs教徒,石を投げないで!}.
でも,人には自作欲求があります.CPU,OS,言語,更に最近はキーボードなどが人気ですが,エディタも中々面白いですよ.
CUIは時代遅れなんてそんなの気にしちゃいけません.
ソースコードは\url{https://github.com/takuzoo3868/td}に置いてあります.

\section{準備}
テキストエディタと言いつつも初めから高機能なエディタを自作するのは至難の業です.
そこで,最低限ファイルを編集して保存できるようにする所から始めるといいかなと思います.
次にシンタックスハイライト対応や文字列検索機能などを考えていきましょう.
リポジトリにあるテキストエディタのファイル構造は以下ようになっています.

\begin{figure}[H]
    \dirtree{%
    .1 td/.
    .2 modules/ \dotfill  \begin{minipage}[t]{7cm}
                              拡張機能を追加していくディレクトリ{.}
    \end{minipage}.
    .3 syntax/ \dotfill  \begin{minipage}[t]{7cm}
                             シンタックスハイライト用の構造体を定義{.}
    \end{minipage}.
    .2 LICENSE.
    .2 Makefile.
    .2 README.md.
    .2 td.c \dotfill  \begin{minipage}[t]{7cm}
                              メインとなるソースコード{.}
    \end{minipage}.
    .2 td.h \dotfill  \begin{minipage}[t]{7cm}
                              定数定義や構造体を含むヘッダ{.}
    \end{minipage}.
    }
\end{figure}
外部ライブラリに依存しない事を目標としていますが,Cコンパイラと\mintinline{bash}{make}コマンドは準備する必要があります.
\mintinline{bash}{cc --version}や\mintinline{bash}{make -v}でインストールされているかどうか確認できます.
自身の環境にコンパイラがインストールされていなかった場合は,Google先生に聞いてみましょう.

\subsubsection{makeによるコンパイル}
解説のために本文中では\mintinline{bash}{hoge}と記載しますが,
好きな名前に置き換えて下さい.
\mintinline{bash}{cc hoge.c -o hoge}などと打ち込めばコンパイルできます.
しかし試行錯誤を繰り返すため,再コンパイルの度に同じ事をするのはあまりスマートではありません.
\mintinline{bash}{make}を用いることでプログラムコンパイルを少しだけ楽にしておきましょう.
\mintinline{bash}{Makefile}を作成し,以下の内容を記述しておきます.
\begin{minted}[frame=lines,framesep=2mm,baselinestretch=1.2,fontsize=\footnotesize,linenos,breaklines]{text}
hoge: hoge.c
$(CC) -o hoge hoge.c -Wall -W -pedantic -std=c99
\end{minted}
この辺については,準備段階なので詳細は省きます.とてもざっくりに言いますと,
諸々の構文をチェックして警告を表示してくれるようオプションを設定しています.これで準備は完了です.

\section{基本構成}
基本となる骨格はkiloというテキストエディタを参考にします\footnote{\url{https://github.com/antirez/kilo}}.
Salvatore Sanfilippo氏によって開発されたC言語製のエディタです.BSD 2-clauseにて公開されています.
紹介文に,
\begin{quote}
Kilo is a small text editor in less than 1K lines of code (counted with cloc).
\end{quote}
とあるように1000行程度なので目で追っていくもの問題ないでしょう.ちょっと厳しいという方は\mintinline{c}{int main()}だけでも目を通す事をお勧めします.
\inputminted[frame=lines,framesep=2mm,baselinestretch=1.2,fontsize=\footnotesize,linenos,breaklines]{c}{\takuzooasset/main.c}
処理の流れはコメントの通り,
\begin{enumerate}
\item 起動にあたりエディタの初期化
\item 引数にあるファイルの拡張子に対応したシンタックスハイライトを適用
\item ファイルをメモリ上へ展開
\item エスケープシーケンスを利用してターミナルをRaw modeへ変更
\item ループ処理
\end{enumerate}
となります.ループでは画面反映とキー入力待ちを行っています.本書では入門編という事でRaw modeの作成を一緒に頑張っていきましょう.

\section{Build your own Editor!!!}
\subsection{Step.1 Raw modeの作成}
ここまでで自作エディタのための開発環境構築は終わっているものとします.
プログラムを書くために,どのエディタを使うかはご自身の信条に従って下さい.
それでは最初の一歩です.以下のコードを書いてみましょう.
\inputminted[frame=lines,framesep=2mm,baselinestretch=1.2,fontsize=\footnotesize,linenos,breaklines]{c}{\takuzooasset/step1_1.c}
\mintinline{c}{unistd.h}から\mintinline{c}{read()}と\mintinline{c}{STDIN_FILENO}を呼び出しています.
\mintinline{c}{read()}は標準入力から1byteを変数\mintinline{c}{c}に読み込んで,読み取るバイトデータがなくなるまで繰り返すようにしてあります.
コンパイルしプログラムを実行すると,端末は標準入力に接続され,キーボードの入力が変数\mintinline{c}{c}に読み込まれます
\footnote{プログラムを終了する場合はCtrl-Dで\mintinline{c}{read()}へ最後へ到達したことを知らせるか,Ctrl-Cでプロセスを終了します.}.
しかし,多くの場合端末は\mintinline{bash}{canonical mode}で起動\footnote{\mintinline{bash}{cooked mode}とも言います}するので,
\mintinline{bash}{raw mode}へ切り替える必要があります.
\mintinline{bash}{canonical mode}はEnterキーを押す事でキーボード入力がプログラムへ渡されます.
テキストエディタの場合は,複雑なインターフェースに加え,キーを押したあとにすぐ処理をしたいので\mintinline{bash}{canonical mode}が適しているとは言えません.
というこで,端末を\mintinline{bash}{raw mode}へ変更しますが,端末内部のフラグをoffにする必要があるので徐々に解説します.

\subsubsection{\mintinline{bash}{ECHO}をoffにする}
端末の\mintinline{bash}{ECHO}機能は入力したキー情報が端末画面に表示され,内容を確認できる優れた機能です.
しかし,レンダリングにおいて\mintinline{bash}{raw mode}では適していないのも事実です.よってoffにしちゃいましょう
\footnote{\mintinline{bash}{sudo}でパスワードを入力するイメージに近いです.}.
\inputminted[frame=lines,framesep=2mm,baselinestretch=1.2,fontsize=\footnotesize,linenos,breaklines]{c}{\takuzooasset/step1_2.c}
手順としては\mintinline{c}{tcgetattr()}を使用して現在の属性を\mintinline{c}{termios}構造体へ読み込み,構造体の変更,
変更された構造体を\mintinline{c}{tcgetattr()}へ渡し,新しい端末属性を書き込むという事をしています.
\mintinline{c}{TCSAFLUSH}は変更をいつ適用するかを指定する引数です.
\mintinline{c}{c_lflag}はローカル用のフラグです.雑多なフラグを管理するためにあります\footnote{macOSの\mintinline{c}{termios.h}には"Local" flags - dumping ground for other stateと書いてあります.}.
そのほか\mintinline{bash}{raw mode}の有効化に関連して変更するフラグは,\mintinline{c}{c_iflag}の入力フラグ,\mintinline{c}{c_oflag}の出力フラグ,\mintinline{c}{c_cflag}の制御フラグです.

次に必要な要素として,プログラムの終了時は端末の元の属性を復元してあげる必要があります.
\mintinline{c}{atexit()}はプログラム終了時に自動的に\mintinline{c}{disableRawMode}を呼び出すために使用します.
端末の元の属性は,\mintinline{c}{orig_termios}構造体に保存しておきます.

\subsubsection{\mintinline{bash}{canonical}のoffとキー入力}
\mintinline{bash}{canonical mode}をoffにするフラグは\mintinline{c}{termios.h}にある\mintinline{c}{ICANON}です.
offにすることで行単位ではなくバイト単位で入力を読み取ることになります.
先程のプログラムの15行目を\mintinline{c}{raw.c_lflag &= ~(ECHO | ICANON);}へ変更し実行してみましょう.
終了時はqキーを押せば大丈夫です.これで\mintinline{bash}{raw mode}へ移行できるようになりました.
それでは入力の様子を知るべく,\mintinline{c}{read()}で読み込んだ各バイトを出力してみましょう.
\inputminted[frame=lines,framesep=2mm,baselinestretch=1.2,fontsize=\footnotesize,linenos,breaklines]{c}{\takuzooasset/step1_3.c}
実行してみると画面にキー入力の結果,どのようにバイト変換されているか表示されるはずです.
\mintinline{c}{iscntrl()}は入力が制御文字かどうか調べてくれます.制御文字とは,画面上に表示できないASCIIコード\footnote{http://www.asciitable.com/}の事を指します.

\subsubsection{出力処理のoffとエラー処理}
ここまで完了したらあとはテキストエディタ用に雑多な処理を記述するだけです.コードは少し長くなります.
\inputminted[frame=lines,framesep=2mm,baselinestretch=1.2,fontsize=\footnotesize,linenos,breaklines]{c}{\takuzooasset/step1_4.c}
Ctrl-C,Ctrl-Z,Ctrl-S,Ctrl-Q,Ctrl-Vを無効化し,Ctrl-Mを\mintinline{bash}{carriage return}へ修正しました.
さらに\mintinline{c}{termios.h}に記載されているいくつかのフラグを無効化しています.

自分が調べた範囲の知識ですが,
\mintinline{c}{BRKINT}はonの状態だとCtrl-Cと同様にSIGINT信号による割り込みを行ってしまいます.
\mintinline{c}{INPCK}は端末におけるデータ整合性確認のためにパリティチェックを可能にします.しかし,現在の端末ではあまり有効な手法ではないため元々offになっている事が多いです.
\mintinline{c}{ISTRIP}は入力されたbyteの第8bitを0へ設定するフラグです.これも大概すでにoffとなっていると思います.
これらのフラグは今現在あるほとんどの端末では既にoffになっている事が多いです.
そんなに気にしなくても良いかなと思いますが,色んな端末に対応したいので設定しておきます.

前段階のソースコードの状態だと\mintinline{c}{read()}はキーボードからの入力を無制限に待ってから復帰するので,タイムアウトを設定しました.
一定時間キーボードから入力がない場合,\mintinline{c}{read()}が返ってきます.設定のために\mintinline{c}{termios.h}から\mintinline{c}{VMIN}と\mintinline{c}{VTIME}を利用しました.
\mintinline{c}{VMIN}は\mintinline{c}{read()}が復帰するまでに必要な最小の入力バイト数を設定します.
ここでは,読み取るべき入力があった場合,すぐに\mintinline{c}{read()}に処理してもらうべく0を設定しています.
\mintinline{c}{VTIME}は\mintinline{c}{read()}へ復帰するまでに待機する最大の時間を設定します.$100 \si{\milli \second}$単位なので注意しましょう.
しかし,WSLの場合\mintinline{c}{VTIME}を無視する事があるようです.今の所,動作に大した影響はないのですが,詳しく調べる必要があるかもしれないです.

続いて,エラー処理についてです.ソースコードにある通り,
\begin{minted}[frame=lines,framesep=2mm,baselinestretch=1.2,fontsize=\footnotesize,linenos,breaklines]{c}
    void die(const char *s) {
        perror(s);
        exit(1);
    }
\end{minted}
にてエラーメッセージを表示してプログラムを終了する関数を設定します.\mintinline{c}{peeor()}は\mintinline{c}{errono}に
設定されているエラーナンバーからエラーメッセージを表示してくれる関数です.よって,この\mintinline{c}{die()}を処理の中に組み込みます.
\mintinline{c}{tcsetattr() tcgetattr() read()}は失敗すると-1を返すようエラーの状態を設定します.

\newpage

エラーを表示できるかの確認として
\mintinline{c}{tcgetattr()}を利用する場合,今の段階ではテキストファイルやパイプを標準入力として与えてみるとよいかもしれないです.
\mintinline{bash}{$ ./hoge <hoge.c}や,\mintinline{bash}{$ echo test | ./hoge}と打ち込むとおそらく次のようなエラーメッセージが返ってくるかなと思います.
\mintinline{bash}{Inappropriate ioctl for device}.現段階ではこれでよしとします.
ちなみにどんなエラーコードがあるのか気になる方は,次のソースコードを実行してみましょう.\mintinline{c}{errono}で定義されているメッセージが表示されるはずです.

\begin{minted}[frame=lines,framesep=2mm,baselinestretch=1.2,fontsize=\footnotesize,linenos,breaklines]{c}
#include <stdio.h>
#include <string.h>
#include <errno.h>
int main(void) {
    int i;
    for (i = 0; i < 132; i++) {
        errno = i;
        printf("%03d : %s\n", errno, strerror(errno));
    }
    return 0;
}
\end{minted}

実行してみると以下のような出力があります.便宜上\mintinline{bash}{error.c}とします.

\begin{minted}[frame=lines,framesep=2mm,baselinestretch=1.2,fontsize=\footnotesize,linenos,breaklines]{bash}
$ gcc -o error error.c
$ ./error
000 : Success
001 : Operation not permitted
002 : No such file or directory
003 : No such process
:
以下略
\end{minted}

\newpage

\section{おわりに}
今回は入門編という事で\mintinline{bash}{raw mode}の実装のみに絞りました.
今後の実装ステップを述べておくと,
\begin{enumerate}
\item Raw modeにおける入出力処理
\item テキストビューワーの作成
\item 編集処理
\item 文字列検索機能
\item シンタックスハイライト
\end{enumerate}
となります.これらの実装手順の解説だけで,一冊の本になりそうな気がします.
自分は現在UTF-8対応や色の設定で悩んでいるので,既存のOSSを読んで勉強している最中です.
今現在対応しているシンタックスハイライトは,C/C++,Python,Brainfuckです.BrainfuckはLOCALの執筆合宿の時にネタで実装してみましたが,見にくい事この上ないです.
何か見やすくなるアイディアをお持ちの方いましたら連絡ください\footnote{https://takuzoo3868.github.io}.

今実装中のものはCUIベースですが,当然中にはVScodeやAtomのようにGUIアプリケーションとして開発したい方もいるでしょう.
その方が需要あるかなと思いましたが,CUIベースは端末の勉強にもなりますし,内部動作を注意深く知っておく必要があるのでかなりオススメです.
最終的には左側でプロジェクトディレクトリなんかも表示させたいな,とかpowerlineに対応したいなとか色々考えています.
やってみると楽しいはずです,ネタ的にも!それでは,よい自作ライフを!!!

\chapterauthor{ゆひ}
\chapter{後で変える}
\input{\articlepath/yuhi}

\chapterauthor{すとんりばー}
\chapter{後で変える}
\section{はじめに}
この章では,同人誌を出すと言われて果たして何を書いたものかと悩んだ情報系学生が乱心し,全く触れたことのないファミコンエミュレータを開発してみた話を書かせていただきます.
ファミコンエミュレータの開発を通して,CPUの動作やリソースの少ない環境での工夫などに触れて行くことを目的としてやっていこうと思います.
色んなサイトから情報を集めつつ書いていきますので,誤った点があるかもしれませんがご了承ください.

\section{エミュレータとは}
一口にエミュレータというと様々なエミュレータが存在しますが,ここで言うエミュレータとは,何らかの機能を持ったコンピュータやハードウェアの動作を模倣するソフトウェアのことです.
コンピュータ分野で馴染みの深いエミュレータとしては,既に起動しているOSの上で他のOSなどのソフトウェアを起動するための仮想環境を提供するソフトウェアなどがあります.

\section{エミュレータの方式}
エミュレータの実行方式には大きく分けて2つの方式が存在します.

まず1つの方式が,エミュレータが動作するハードウェアの実CPUの計算能力の一部をそのまま利用する方式です.この方式ではエミュレータ上で動作させるソフトウェアへのオーバーヘッドが少なく,高速な動作が可能です.
しかし,この方式では動作させたいソフトウェアが想定するアーキテクチャが利用する実ハードウェアと一致していなければならないというデメリットがあります.
この方式にはVM VirtualBoxやVMWare Playerなどが該当し,これらのエミュレータの上では,前述の理由からARMやRICS-Vなどに向けたソフトウェアが動作しません.

もう一つの方式として,特定のアーキテクチャのCPUの動作をソフトウェア上で模倣し,その上で動作させたいソフトウェアをエミュレートする方式があります.
こちらの方式では,エミュレートしたいCPUのアーキテクチャに従って実装を行えば実ハードウェアのアーキテクチャは関係がないため,より多くのソフトウェアをエミュレートすることができます.
しかし,実ハードウェアの上で別のCPUの動作をするソフトウェアを動かすということになるため,エミュレートによるオーバーヘッドが大きくなります.
この方式にはQEMUなどが該当し,Android Studioなどに搭載されているAndroidエミュレータもQEMUをベースに開発がなされているため,ARMアーキテクチャ向けであるAndroidをx86環境上で動作させることができています.\footnote{Android-x86などのx86移植版はVirtualBoxやVMWareでも動作しますが,一部のAndroidアプリケーションの動作に違いが出ることがあります.}
また,こちらの方式には高レベルエミュレーションと低レベルエミュレーションと呼ばれる更に細かい2つの分類が存在します.
それぞれ,エミュレートするCPUをハードウェアレベルからソフトウェアで実装する方式と,命令に対する応答のみをソフトウェアで実装する方式となっており,より精密なエミュレーションが要求される場合は前者,動作のみをエミュレートすれば良い場合は後者が選ばれます.一般的に,高レベルエミュレーションのほうが軽量に動作します.

\section{ゲームエミュレータとは}
ゲームエミュレータとは,特定のコンシューマ(コンソール)機やアーケード機などのゲームハードに向けて開発されたソフトウェアを,実機以外の環境で動作させることを目的としたエミュレータのことです.
実際にメーカーが開発したゲームソフトウェアをエミュレータで動作させる行為については色々と権利問題が複雑なためここでは触れませんが,個人がコンシューマー機などに向けて開発したオリジナルのソフトウェア(ROM)も多く存在し,もちろんそれらは安全にエミュレータ上で動作させることが可能です.
自分でROMを作成したオリジナルのゲームを名のしれたゲームハードと同じ環境で動かすことができるのってなかなか夢がありませんか?

また,任天堂のWiiに搭載されているバーチャルコンソールなどは,過去に自社ハードで発売したゲームをWiiで動作させる公式のゲームエミュレータとして開発されたものです.

\section{ファミコンの構成}
今回の開発では,ファミコンことファミリーコンピュータのエミュレータを開発してみますが,その前にまずはファミコンがどのような構成で動作しているのか確認しましょう.なお,ここで扱う構成は日本で発売されたファミコンを対象とし,更にディスクシステムやネットワークシステムについては扱いません.

\subsection{全体の概要}
まずは,全体の概要を見てみます.表\ref{famicom-spec}におおまかなファミコンのスペックをまとめてみました.
\begin{table}[h]
\centering
\caption{スペック概要}
\label{famicom-spec}
\begin{tabular}{|l|l|l|}
\hline
\multirow{3}{*}{CPU} & チップ & Ricoh 2A03 (MOS 6502カスタム) \\ \cline{2-3}
 & アドレス空間 & 64KB (\$0000 - \$FFFF) \\ \cline{2-3}
 & クロック & 1.789773 MHz \\ \hline
Main Memory & ワーキングRAM & 2KB \\ \hline
\multirow{5}{*}{Graphics} & チップ & Ricoh 2C02 \\ \cline{2-3}
 & ビデオRAM & 2KB \\ \cline{2-3}
 & 解像度 & 256x240 (W x H) オーバースキャン非考慮 \\ \cline{2-3}
 & 色数 & 52色 (13色x4階調) \\ \cline{2-3}
 & 同時色数 & \begin{tabular}[c]{@{}l@{}}BG色13色(汎用12色+背景色),\\ スプライト12色(汎用12色,背景は透明な為無し),\\ 計25色\end{tabular} \\ \hline
\multirow{2}{*}{Sounds} & チップ & Ricoh 2A03 内蔵機能 \\ \cline{2-3}
 &  & あとでかく \\ \hline
\multirow{2}{*}{カートリッジ} &  & あとでかく \\ \cline{2-3}
 &  &  \\ \hline
\end{tabular}
\end{table}


\section{ファミコンのスペック}
ファミコンの各種スペックを以下に示します.
\begin{minted}[frame=lines,framesep=2mm,baselinestretch=1.2,fontsize=\footnotesize,linenos,breaklines]{text}
\end{minted}

\section{検証環境}
今回開発を行うにあたって使用した環境を以下に示します.
\begin{minted}[frame=lines,framesep=2mm,baselinestretch=1.2,fontsize=\footnotesize,linenos,breaklines]{text}
    Archlinux (x86_64 5.0.3.arch1-1)
    go 1.12 linux/amd64
    sdl 2.0.9
\end{minted}




\newpage
\myimpression[%
name=LOCAL Students\\情報ボーイズの寄稿ノート, %
author=うっひょい, \and %
ちくうぇいと, \and %
あわあわ, \\ \and %
けんつ, \and %
さわだ, \and %
Jumpaku, \and %
あるねこ, %
date=2018年4月22日, %
publisher=LOCAL学生部, %
print=有限会社ねこのしっぽ %
]%
\end{document}
