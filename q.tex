\contentsline {chapter}{はじめに}{iv}{chapter*.1}
\contentsline {chapter}{\numberline {第1章}\LaTeX の乱数生成アルゴリズムを改善する}{1}{chapter.1}
\vskip -10pt
\contentsline {chapter}{\hskip 1.3em\mdseries \scriptsize うっひょい}{}{}
\vskip 5pt
\contentsline {section}{\numberline {1.1}乱数とは}{1}{section.1.1}
\contentsline {subsection}{\numberline {1.1.1}疑似乱数と真の乱数}{1}{subsection.1.1.1}
\contentsline {subsection}{\numberline {1.1.2}準乱数とは}{1}{subsection.1.1.2}
\contentsline {section}{\numberline {1.2}乱数生成コマンド}{1}{section.1.2}
\contentsline {subsection}{\numberline {1.2.1}固定小数演算}{1}{subsection.1.2.1}
\contentsline {subsection}{\numberline {1.2.2}乱数を出力するFPrandomコマンド}{2}{subsection.1.2.2}
\contentsline {section}{\numberline {1.3}FPrandomの乱数生成アルゴリズムを調べる}{2}{section.1.3}
\contentsline {subsection}{\numberline {1.3.1}目的}{2}{subsection.1.3.1}
\contentsline {subsection}{\numberline {1.3.2}疑似乱数アルゴリズムの問題点}{2}{subsection.1.3.2}
\contentsline {subsection}{\numberline {1.3.3}ソースを読む}{3}{subsection.1.3.3}
\contentsline {subsubsection}{コメントに正解が書いてあった}{3}{subsubsection*.3}
\contentsline {subsubsection}{T\kern -.1667em\lower .5ex\hbox {E}\kern -.125emX\bxjs@SE {}のマクロで実装された疑似乱数アルゴリズム}{3}{subsubsection*.4}
\contentsline {subsubsection}{実際に出力してみる}{5}{subsubsection*.5}
\contentsline {chapter}{\numberline {第2章}超入門 仮想化技術}{7}{chapter.2}
\vskip -10pt
\contentsline {chapter}{\hskip 1.3em\mdseries \scriptsize chikuwait(ちくうぇいと)}{}{}
\vskip 5pt
\contentsline {chapter}{\numberline {第3章}ほげ}{8}{chapter.3}
\vskip -10pt
\contentsline {chapter}{\hskip 1.3em\mdseries \scriptsize あわあわ}{}{}
\vskip 5pt
\contentsline {chapter}{\numberline {第4章}入門Linux Kernel}{9}{chapter.4}
\vskip -10pt
\contentsline {chapter}{\hskip 1.3em\mdseries \scriptsize けんつ}{}{}
\vskip 5pt
\contentsline {section}{\numberline {4.1}はじめに}{9}{section.4.1}
\contentsline {subsection}{\numberline {4.1.1}カーネルモジュールとは}{9}{subsection.4.1.1}
\contentsline {subsection}{\numberline {4.1.2}カーネルからHello,World!!}{10}{subsection.4.1.2}
\contentsline {subsection}{\numberline {4.1.3}}{10}{subsection.4.1.3}
\contentsline {chapter}{\numberline {第5章}自作エディタ入門編}{11}{chapter.5}
\vskip -10pt
\contentsline {chapter}{\hskip 1.3em\mdseries \scriptsize さわだ}{}{}
\vskip 5pt
\contentsline {section}{\numberline {5.1}はじめに}{11}{section.5.1}
\contentsline {section}{\numberline {5.2}準備}{11}{section.5.2}
\contentsline {subsubsection}{makeによるコンパイル}{12}{subsubsection*.6}
\contentsline {section}{\numberline {5.3}基本構成}{13}{section.5.3}
\contentsline {section}{\numberline {5.4}Build your own Editor!!!}{20}{section.5.4}
\contentsline {subsection}{\numberline {5.4.1}シンタックスハイライト対応}{20}{subsection.5.4.1}
\contentsline {chapter}{\numberline {第6章}世界と孤独の\jruby [g]{説法}{エピローグ}}{21}{chapter.6}
\vskip -10pt
\contentsline {chapter}{\hskip 1.3em\mdseries \scriptsize Jumpaku}{}{}
\vskip 5pt
\contentsline {chapter}{\numberline {第7章}関数型Python入門}{24}{chapter.7}
\vskip -10pt
\contentsline {chapter}{\hskip 1.3em\mdseries \scriptsize あるねこ}{}{}
\vskip 5pt
\contentsline {section}{\numberline {7.1}Pythonと型}{24}{section.7.1}
\contentsline {section}{\numberline {7.2}リストを改造しよう}{24}{section.7.2}
\contentsline {section}{\numberline {7.3}Maybeを作り出す}{24}{section.7.3}
tentsline {section}{\numberline {4.1}はじめに}{9}{section.4.1}}
\@writefile{toc}{\contentsline {subsection}{\numberline {4.1.1}カーネルモジュールとは}{9}{subsection.4.1.1}}
\@writefile{toc}{\contentsline {subsection}{\numberline {4.1.2}カーネルからHello,World!!}{10}{subsection.4.1.2}}
\@writefile{toc}{\contentsline {subsection}{\numberline {4.1.3}}{10}{subsection.4.1.3}}
\@writefile{toc}{\contentsline {chapter}{\numberline {第5章}自作エディタ入門編}{11}{chapter.5}}
\@writefile{lof}{\addvspace {10\jsc@mpt }}
\@writefile{lot}{\addvspace {10\jsc@mpt }}
\@writefile{toc}{\vskip -10pt}
\@writefile{toc}{\contentsline {chapter}{\hskip 1.3em\mdseries  \scriptsize さわだ}{}{}}
\@writefile{toc}{\vskip 5pt}
\@writefile{toc}{\contentsline {section}{\numberline {5.1}はじめに}{11}{section.5.1}}
\@writefile{toc}{\contentsline {section}{\numberline {5.2}準備}{11}{section.5.2}}
\@writefile{toc}{\contentsline {subsubsection}{makeによるコンパイル}{12}{subsubsection*.6}}
\@writefile{toc}{\contentsline {section}{\numberline {5.3}基本構成}{13}{section.5.3}}
\@writefile{toc}{\contentsline {section}{\numberline {5.4}Build your own Editor!!!}{20}{section.5.4}}
\@writefile{toc}{\contentsline {subsection}{\numberline {5.4.1}シンタックスハイライト対応}{20}{subsection.5.4.1}}
\@writefile{toc}{\contentsline {chapter}{\numberline {第6章}世界と孤独の\jruby  [g]{説法}{エピローグ}}{21}{chapter.6}}
\@writefile{lof}{\addvspace {10\jsc@mpt }}
\@writefile{lot}{\addvspace {10\jsc@mpt }}
\@writefile{toc}{\vskip -10pt}
\@writefile{toc}{\contentsline {chapter}{\hskip 1.3em\mdseries  \scriptsize Jumpaku}{}{}}
\@writefile{toc}{\vskip 5pt}
\@writefile{toc}{\contentsline {chapter}{\numberline {第7章}関数型Python入門}{24}{chapter.7}}
\@writefile{lof}{\addvspace {10\jsc@mpt }}
\@writefile{lot}{\addvspace {10\jsc@mpt }}
\@writefile{toc}{\vskip -10pt}
\@writefile{toc}{\contentsline {chapter}{\hskip 1.3em\mdseries  \scriptsize あるねこ}{}{}}
\@writefile{toc}{\vskip 5pt}
\@writefile{toc}{\contentsline {section}{\numberline {7.1}はじめに}{24}{section.7.1}}
\@writefile{toc}{\contentsline {section}{\numberline {7.2}型とPythonとmypyと}{24}{section.7.2}}
\@writefile{toc}{\contentsline {subsection}{\numberline {7.2.1}Pythonにおける型ヒンティング}{24}{subsection.7.2.1}}
\@writefile{toc}{\contentsline {subsection}{\numberline {7.2.2}mypyによる型の解析}{25}{subsection.7.2.2}}
\@writefile{toc}{\contentsline {section}{\numberline {7.3}型クラスを作る}{27}{section.7.3}}
\@writefile{toc}{\contentsline {subsection}{\numberline {7.3.1}型クラス概要}{27}{subsection.7.3.1}}
\@writefile{toc}{\contentsline {subsection}{\numberline {7.3.2}Functorの実装}{27}{subsection.7.3.2}}
\@writefile{toc}{\contentsline {section}{\numberline {7.4}List Monadを作る}{28}{section.7.4}}
\@writefile{toc}{\contentsline {section}{\numberline {7.5}Maybeを作り出す}{28}{section.7.5}}
